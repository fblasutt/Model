\documentclass[]{article}

\usepackage[margin=2cm]{geometry}
\usepackage{graphicx} % Allows including images
\usepackage{booktabs} % Allows the use of \toprule, \midrule and \bottomrule in tables
\usepackage{textcomp}
\usepackage{threeparttable}
\usepackage{tikz}
\usepackage[utf8]{inputenc}
\usepackage{makecell}
\usepackage{marvosym}
\usepackage{eurosym}
\usepackage{threeparttablex}
\usepackage{tabularx}
\DeclareUnicodeCharacter{20AC}{\EUR{}}
\usepackage[normalem]{ulem}

%opening
\title{Family Risk Sharing}
\author{B-C-V}

\begin{document}

\maketitle

%%%%%%%%%%%%%%%%%%%%%%%%
%All income
%%%%%%%%%%%%%%%%%%%%%%%
\begin{table}[h]\centering
	
	\caption{Pass-through of changes in income on consumption and consumption shares, using changes in...}
	\label{table:allinc}
	\begin{threeparttable}[t]\centering
		\begin{tabular*}{\textwidth}{l@{\extracolsep{\textwidth minus \textwidth}}ccccc}
			\toprule
			& Total Exp  & Common Exp  & Husband Exp & Wife Exp & Wife share  \\[0.5ex]
			&  (1)& (2) & (3) & (4) & (5)   \\[0.5ex]
			\midrule		
			...total income   & \textbf{0.404} & 0.367 & & &    \\ ...wife income    & 0.170 & 0.165&  \textbf{0.143} &  \textbf{0.227} &  \textbf{0.084}    \\ ...husband income & 0.194 &  0.191&  \textbf{0.215} &  \textbf{0.184} &  \textbf{-0.031}    \\\bottomrule    
			\\[-2.5ex] 
		\end{tabular*}
		\begin{tablenotes}[flushleft]
			\footnotesize{\item \textsc{Notes}: Coefficient interpretation: 1\% change in income leads to X\% change in expenditure. Coefficients associated to changes in the wife income are computed using women working in two consecutive periods.
			}
		\end{tablenotes}
	\end{threeparttable}
\end{table}

%%%%%%%%%%%%%%%%%%%%%%%%
%Transitory income changes
%%%%%%%%%%%%%%%%%%%%%%%
\begin{table}[h]\centering
	
	\caption{Pass-through of changes in income on consumption and consumption shares, using \textbf{transitory
		} changes in...}
	\label{table:trainc}
	\begin{threeparttable}[t]\centering
		\begin{tabular*}{\textwidth}{l@{\extracolsep{\textwidth minus \textwidth}}ccccc}
			\toprule
			& Total Exp  & Common Exp  & Husband Exp & Wife Exp & Wife share  \\[0.5ex]
			&  (1)& (2) & (3) & (4) & (5)   \\[0.5ex]
			\midrule		
			...total income   & \textbf{0.054} & 0.051 & & &    \\ ...wife income    & 0.040 & 0.039&  \textbf{0.045} &  \textbf{0.044} &  \textbf{0.000}    \\ ...husband income & 0.054 &  0.052&  \textbf{0.072} &  \textbf{0.013} &  \textbf{-0.048}    \\\bottomrule    
			\\[-2.5ex] 
		\end{tabular*}
		\begin{tablenotes}[flushleft]
			\footnotesize{\item \textsc{Notes}: Coefficient interpretation: 1\% change in income leads to X\% change in expenditure. Coefficients associated to changes in the wife income are computed using women working in two consecutive periods.
			}
		\end{tablenotes}
	\end{threeparttable}
\end{table}

%%%%%%%%%%%%%%%%%%%%%%%%%%%
%Persistent income changes
%%%%%%%%%%%%%%%%%%%%%%%%%%
\begin{table}[h]\centering
	
	\caption{Pass-through of changes in income on consumption and consumption shares, using \textbf{persistent} changes in...}
	\label{table:persinc}
	\begin{threeparttable}[t]\centering
		\begin{tabular*}{\textwidth}{l@{\extracolsep{\textwidth minus \textwidth}}ccccc}
			\toprule
			& Total Exp  & Common Exp  & Husband Exp & Wife Exp & Wife share  \\[0.5ex]
			&  (1)& (2) & (3) & (4) & (5)   \\[0.5ex]
			\midrule		
			...total income   & \textbf{0.321} & 0.308 & & &    \\ ...wife income    & 0.346 & 0.339&  \textbf{0.313} &  \textbf{0.504} &  \textbf{0.191}    \\ ...husband income & 0.330 &  0.322&  \textbf{0.390} &  \textbf{0.322} &  \textbf{-0.068}    \\\bottomrule    
			\\[-2.5ex] 
		\end{tabular*}
		\begin{tablenotes}[flushleft]
			\footnotesize{\item \textsc{Notes}: Coefficient interpretation: 1\% change in income leads to X\% change in expenditure. Coefficients associated to changes in the wife income are computed using women working in two consecutive periods.
			}
		\end{tablenotes}
	\end{threeparttable}
\end{table}

%%%%%%%%%%%%%%%%%%%%%%%%
%BPP MPC
%%%%%%%%%%%%%%%%%%%%%%%
\begin{table}[h]\centering
	
	\caption{MPC calculated as in BPP, using transitory changes in...}
	\label{table:BPP_MPC}
	\begin{threeparttable}[t]\centering
		\begin{tabular*}{\textwidth}{l@{\extracolsep{\textwidth minus \textwidth}}cccc}
			\toprule
			& Total Exp  & Common Exp  & Husband Exp & Wife Exp  \\[0.5ex]
			&  (1)& (2) & (3) & (4)   \\[0.5ex]
			\midrule		
			...husband income & 0.031 & 0.030 & 0.037 & -0.011  \\ ...wife income    & 0.018 & 0.019 & 0.007 & 0.062  \\ ...total income   & 0.268 & 0.234 & 0.404 & 0.416  \\\bottomrule    
			\\[-2.5ex] 
		\end{tabular*}
		\begin{tablenotes}[flushleft]
			\footnotesize{\item \textsc{Notes}: the consumption insurance parameters displayed in the table are computed as $$\frac{E\left(\Delta c_t \Delta y_{t+1}\right)}{E\left(\Delta y_t \Delta y_{t+1}\right)},$$ where $y_t$ can the income of the husband, wife or the sum of the two (total). Variables $c_t$ can be the total, common, husband or wife' expenditures. Coefficients associated to changes in the wife income are computed using women working in two consecutive periods.
			}
		\end{tablenotes}
	\end{threeparttable}
\end{table}


%%%%%%%%%%%%%%%%%%%%%%%%
%BPP PERSISTENT
%%%%%%%%%%%%%%%%%%%%%%%
\begin{table}[h]\centering
	
	\caption{Consumption insurance to persistent income shocks, calculated as in BPP, using persistent changes in...}
	\label{table:BPP_PER}
	\begin{threeparttable}[t]\centering
		\begin{tabular*}{\textwidth}{l@{\extracolsep{\textwidth minus \textwidth}}cccc}
			\toprule
			& Total Exp  & Common Exp  & Husband Exp & Wife Exp  \\[0.5ex]
			&  (1)& (2) & (3) & (4)   \\[0.5ex]
			\midrule		
			...husband income & 0.370 & 0.365 & 0.418 & 0.354  \\ ...wife income    & 0.431 & 0.420 & 0.379 & 0.598  \\ ...total income   & 0.541 & 0.516 & 0.636 & 0.690  \\\bottomrule     
			\\[-2.5ex] 
		\end{tabular*}
		\begin{tablenotes}[flushleft]
			\footnotesize{\item \textsc{Notes}: the consumption insurance parameters displayed in the table are computed as $$\frac{E\left(\Delta c_t\left(\Delta y_{t-1}+\Delta y_t+\Delta y_t\right)\right)}{E\left(\Delta y_t\left(\Delta y_{t-1}+\Delta y_t+\Delta y_t\right)\right)},$$ where $y_t$ can the income of the husband, wife or the sum of the two (total). Variables $c_t$ can be the total, common, husband or wife' expenditures. Coefficients associated to changes in the wife income are computed using women working in two consecutive periods.
}
		\end{tablenotes}
	\end{threeparttable}
\end{table}


\end{document}
