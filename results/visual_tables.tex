\documentclass[]{article}

\usepackage[margin=2cm]{geometry}
\usepackage{graphicx} % Allows including images
\usepackage{booktabs} % Allows the use of \toprule, \midrule and \bottomrule in tables
\usepackage{textcomp}
\usepackage{threeparttable}
\usepackage{tikz}
\usepackage[utf8]{inputenc}
\usepackage{makecell}
\usepackage{marvosym}
\usepackage{eurosym}
\usepackage{threeparttablex}
\usepackage{tabularx}
\DeclareUnicodeCharacter{20AC}{\EUR{}}
\usepackage[normalem]{ulem}

%opening
\title{Family Risk Sharing}
\author{B-C-V}

\begin{document}

\maketitle

%%%%%%%%%%%%%%%%%%%%%%%%
%All income
%%%%%%%%%%%%%%%%%%%%%%%
\begin{table}[h]\centering
	
	\caption{Pass-through of changes in income on consumption and consumption shares, using changes in...}
	\label{table:allinc}
	\begin{threeparttable}[t]\centering
		\begin{tabular*}{\textwidth}{l@{\extracolsep{\textwidth minus \textwidth}}ccccc}
			\toprule
			& Total Exp  & Common Exp  & Husband Exp & Wife Exp & Wife share  \\[0.5ex]
			&  (1)& (2) & (3) & (4) & (5)   \\[0.5ex]
			\midrule		
			...total income   & \textbf{0.309} & 0.276 & & &    \\ ...wife income    & 0.080 & 0.077&  \textbf{0.092} &  \textbf{0.096} &  \textbf{0.004}    \\ ...husband income & 0.177 &  0.172&  \textbf{0.204} &  \textbf{0.179} &  \textbf{-0.025}    \\\bottomrule    
			\\[-2.5ex] 
		\end{tabular*}
		\begin{tablenotes}[flushleft]
			\footnotesize{\item \textsc{Notes}: Coefficient interpretation: 1\% change in income leads to X\% change in expenditure. Coefficients associated to changes in the wife income are computed using women working in two consecutive periods.
			}
		\end{tablenotes}
	\end{threeparttable}
\end{table}



%%%%%%%%%%%%%%%%%%%%%%%%
%Transitory income changes
%%%%%%%%%%%%%%%%%%%%%%%
\begin{table}[h]\centering
	
	\caption{Pass-through of changes in income on consumption and consumption shares, using \textbf{transitory
		} changes in...}
	\label{table:trainc}
	\begin{threeparttable}[t]\centering
		\begin{tabular*}{\textwidth}{l@{\extracolsep{\textwidth minus \textwidth}}ccccc}
			\toprule
			& Total Exp  & Common Exp  & Husband Exp & Wife Exp & Wife share  \\[0.5ex]
			&  (1)& (2) & (3) & (4) & (5)   \\[0.5ex]
			\midrule		
			...total income   & \textbf{0.054} & 0.051 & & &    \\ ...wife income    & 0.040 & 0.039&  \textbf{0.045} &  \textbf{0.044} &  \textbf{0.000}    \\ ...husband income & 0.054 &  0.052&  \textbf{0.072} &  \textbf{0.013} &  \textbf{-0.048}    \\\bottomrule    
			\\[-2.5ex] 
		\end{tabular*}
		\begin{tablenotes}[flushleft]
			\footnotesize{\item \textsc{Notes}: Coefficient interpretation: 1\% change in income leads to X\% change in expenditure. Coefficients associated to changes in the wife income are computed using women working in two consecutive periods.
			}
		\end{tablenotes}
	\end{threeparttable}
\end{table}

%%%%%%%%%%%%%%%%%%%%%%%%%%%
%Persistent income changes
%%%%%%%%%%%%%%%%%%%%%%%%%%
\begin{table}[h]\centering
	
	\caption{Pass-through of changes in income on consumption and consumption shares, using \textbf{persistent} changes in...}
	\label{table:persinc}
	\begin{threeparttable}[t]\centering
		\begin{tabular*}{\textwidth}{l@{\extracolsep{\textwidth minus \textwidth}}ccccc}
			\toprule
			& Total Exp  & Common Exp  & Husband Exp & Wife Exp & Wife share  \\[0.5ex]
			&  (1)& (2) & (3) & (4) & (5)   \\[0.5ex]
			\midrule		
			...total income   & \textbf{0.303} & 0.330 & & &    \\ ...wife income    & 0.127 & 0.136&  \textbf{0.033} &  \textbf{0.084} &  \textbf{0.051}    \\ ...husband income & 0.412 &  0.435&  \textbf{0.240} &  \textbf{0.226} &  \textbf{-0.014}    \\\bottomrule    
			\\[-2.5ex] 
		\end{tabular*}
		\begin{tablenotes}[flushleft]
			\footnotesize{\item \textsc{Notes}: Coefficient interpretation: 1\% change in income leads to X\% change in expenditure. Coefficients associated to changes in the wife income are computed using women working in two consecutive periods.
			}
		\end{tablenotes}
	\end{threeparttable}
\end{table}

%%%%%%%%%%%%%%%%%%%%%%%%
%BPP MPC
%%%%%%%%%%%%%%%%%%%%%%%
\begin{table}[h]\centering
	
	\caption{MPC calculated as in BPP, using transitory changes in...}
	\label{table:BPP_MPC}
	\begin{threeparttable}[t]\centering
		\begin{tabular*}{\textwidth}{l@{\extracolsep{\textwidth minus \textwidth}}cccc}
			\toprule
			& Total Exp  & Common Exp  & Husband Exp & Wife Exp  \\[0.5ex]
			&  (1)& (2) & (3) & (4)   \\[0.5ex]
			\midrule		
			...husband income & 0.039 & 0.041 & 0.044 & 0.001  \\ ...wife income    & 0.041 & 0.039 & -0.012 & 0.083  \\ ...total income   & 0.319 & 0.279 & 0.473 & 0.521  \\\bottomrule    
			\\[-2.5ex] 
		\end{tabular*}
		\begin{tablenotes}[flushleft]
			\footnotesize{\item \textsc{Notes}: the consumption insurance parameters displayed in the table are computed as $$\frac{E\left(\Delta c_t \Delta y_{t+1}\right)}{E\left(\Delta y_t \Delta y_{t+1}\right)},$$ where $y_t$ can the income of the husband, wife or the sum of the two (total). Variables $c_t$ can be the total, common, husband or wife' expenditures. Coefficients associated to changes in the wife income are computed using women working in two consecutive periods.
			}
		\end{tablenotes}
	\end{threeparttable}
\end{table}


%%%%%%%%%%%%%%%%%%%%%%%%
%BPP PERSISTENT
%%%%%%%%%%%%%%%%%%%%%%%
\begin{table}[h]\centering
	
	\caption{Consumption insurance to persistent income shocks, calculated as in BPP, using persistent changes in...}
	\label{table:BPP_PER}
	\begin{threeparttable}[t]\centering
		\begin{tabular*}{\textwidth}{l@{\extracolsep{\textwidth minus \textwidth}}cccc}
			\toprule
			& Total Exp  & Common Exp  & Husband Exp & Wife Exp  \\[0.5ex]
			&  (1)& (2) & (3) & (4)   \\[0.5ex]
			\midrule		
			...husband income & 0.364 & 0.352 & 0.426 & 0.346  \\ ...wife income    & 0.341 & 0.334 & 0.310 & 0.499  \\ ...total income   & 0.440 & 0.418 & 0.507 & 0.619  \\\bottomrule     
			\\[-2.5ex] 
		\end{tabular*}
		\begin{tablenotes}[flushleft]
			\footnotesize{\item \textsc{Notes}: the consumption insurance parameters displayed in the table are computed as $$\frac{E\left(\Delta c_t\left(\Delta y_{t-1}+\Delta y_t+\Delta y_t\right)\right)}{E\left(\Delta y_t\left(\Delta y_{t-1}+\Delta y_t+\Delta y_t\right)\right)},$$ where $y_t$ can the income of the husband, wife or the sum of the two (total). Variables $c_t$ can be the total, common, husband or wife' expenditures. Coefficients associated to changes in the wife income are computed using women working in two consecutive periods.
}
		\end{tablenotes}
	\end{threeparttable}
\end{table}

%%%%%%%%%%%%%%%%%%%%%%%%%%%%%%%%%%
%Income shocks and labor supply
%%%%%%%%%%%%%%%%%%%%%%%%%%%%%%%%%
\begin{table}[h]\centering
	
	\caption{Women's employment response (in percentage points) to different types of income shocks}
	\label{table:shocks_WLP}
	\begin{threeparttable}[t]\centering
		\begin{tabular*}{\textwidth}{@{\extracolsep{\textwidth minus \textwidth}}cccccc}
			\toprule
			  \multicolumn{2}{c}{Transitory shocks}  &  \multicolumn{2}{c}{Persistent shocks} & \multicolumn{2}{c}{Transitory+persistent shocks}   \\[0.5ex]
			
		    \cmidrule(lr){1-2}    \cmidrule(lr){3-4}  \cmidrule(lr){5-6} 
			
			
			
			 Wife  & Husband  &Wife  & Husband  & Wife  & Husband     \\[0.5ex]
			  (1)& (2) & (3) & (4) & (5)  & (6)  \\[0.5ex]
			\midrule		
			  0.337 & -0.033 & 0.403 & -0.055 & 0.358 & -0.042  \\\bottomrule     
			\\[-2.5ex] 
		\end{tabular*}
		\begin{tablenotes}[flushleft]
			\footnotesize{\item \textsc{Notes}: the income shocks relate to \textit{potential log income} $y$. In the case of women, a positive potential income shocks does not translate in more earnings if the women does not work. The numbers displayed in the table are OLS coefficients:
				
				$$\frac{E(\Delta y_t \  \Delta WLP_t)}{E(\Delta y_t)},$$
			
			where $\Delta WLP$ is the change in women's employment over two consecutive periods.
				
			}
		\end{tablenotes}
	\end{threeparttable}
\end{table}

%%%%%%%%%%%%%%%%%%%%%%%%
%Level
%%%%%%%%%%%%%%%%%%%%%%%
\begin{table}[h]\centering
	
	\caption{Pass-through of changes in income on consumption and consumption shares, using changes in...}
	\label{table:level}
	\begin{threeparttable}[t]\centering
		\begin{tabular*}{\textwidth}{l@{\extracolsep{\textwidth minus \textwidth}}ccccc}
			\toprule
			& Total Exp  & Common Exp  & Husband Exp & Wife Exp & Wife share  \\[0.5ex]
			&  (1)& (2) & (3) & (4) & (5)   \\[0.5ex]
			\midrule		
			...total income   & \textbf{0.173} & 0.116 & & &    \\ ...wife income    & 0.171 & 0.113&  \textbf{0.035} &  \textbf{0.023} &  \textbf{0.013}    \\ ...husband income & 0.152 &  0.117&  \textbf{0.025} &  \textbf{0.011} &  \textbf{-0.016}    \\\bottomrule    
			\\[-2.5ex] 
		\end{tabular*}
		\begin{tablenotes}[flushleft]
			\footnotesize{\item \textsc{Notes}: Coefficient interpretation: 1 yen change in income leads to X yen change in expenditure. 
			}
		\end{tablenotes}
	\end{threeparttable}
\end{table}
\end{document}
